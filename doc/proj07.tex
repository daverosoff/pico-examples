\begin{document}

In Project 06 you created a new CMake project, \texttt{fib_display}, and now
we'd like to extract some of its functionality into a separate library that can
be used in other projects.  This is a common pattern in software development,
and indeed you did it yourself in Project 02 in the CircuitPython context.

\begin{minipage}{0.8\textwidth}[t]
    \raggedright
    \emph{Languages don't matter. Concepts matter. When you learn a programming
    concept, you can recognize it or its equivalent in any language.}
\end{minipage}

\section*{Before you begin\ldots}
Keep the best practices in mind:
\begin{enumerate}
    \item Always open the top-level folder, the \texttt{pico-examples} folder,
    with VSCode. Remember to use the \texttt{code .} command from the
    Developers Console if your build setup requires it.
    \item If it seems like your build process is messed up, do
    \texttt{Ctrl+\langle comma\rangle} and check that settings \texttt{cmake.generator}
    and \texttt{cmake.}
\end{enumerate}

\section*{Part A: Create the folder and \texttt{CMakeLists.txt}}

% Let's go, Copilot!
Create a new folder called \texttt{fib} in the \texttt{pico-examples} folder. Inside,
create the header file \texttt{fib.h} and the source file \texttt{fib.c}.  In
the same folder, create a new file called \texttt{CMakeLists.txt}. The \texttt{CMakeLists.txt}
should look like this:

\begin{lstlisting}[language=CMake]
add_library(fib fib.c)
# target_include_directories(fib PUBLIC .)
target_link_libraries(fib pico_stdlib)
pico_add_extra_outputs(fib)
example_auto_set_url(fib)
\end{lstlisting}

Next, paste the \emph{declarations} from the \texttt{fib_display.h} file into
the \texttt{fib.h} file.  You can delete them from the \texttt{fib_display.h} file.
The \texttt{fib.h} file should look like this:
\begin{lstlisting}[language=C]
